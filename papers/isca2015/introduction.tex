\section{Introduction}
\label{sec:intro}
By many accounts, complex analysis of Big Data is going to be the biggest
economic driver for the IT industry. For example, Google has predicted flu
outbreaks by analyzing social network information a week faster than
CDC~\cite{googleflu}; Analysis of twitter data can reveal social upheavals
faster than journalists; Amazon is planning to use customer data for
anticipatory
shipping of products~\cite{amazonanticipatory}; Real-time analysis of personal genome may significantly
aid in diagnostics. Big Data analytics are potentially going to have
revolutionary impact on the way scientific discoveries are made. 

Big Data by definition doesn’t fit in personal computers or DRAM of even
moderate size clusters. Since the data may be stored on hard disks, latency and
throughput of storage access is of primary concern. Historically, this has been
mitigated by organizing the processing of data in a highly sequential manner.
However, complex queries cannot always be organized for sequential data
accesses, and thus high performance implementations of such queries pose a great
challenge. One approach to solving this problem is \emph{ram
cloud}~\cite{ramcloud}, where the cluster has enough collective DRAM to
accommodate the entire dataset in DRAM.  In this paper, we explore a much
cheaper alternative where Big Data analytics can be done with reasonable
efficiency in a single rack with distributed flash storage, which has much
better random accesses performance than hard disks.  We call our system
FlashBoost and it provides the following capabilities:

\begin{enumerate}
\item A-20 node system with large enough flash storage to host Big Data workloads in the 10\textasciitilde20 TB range;
\item Near-uniform latency access into a network of storage devices that form a global address space;
\item Capacity to implement user-defined in-storage processing engines;
\item Flash card design which exposes an interface to make application-specific optimizations in flash accesses.
\end{enumerate}

Our preliminary experimental results show that for some applications, FlashBoost
performance is an order of magnitude better than a conventional cluster where
SSDs are used only as a disk replacement. FlashBoost unambiguously establishes
an architecture whose price-performance-power characteristics provide an
attractive alternative for doing similar scale applications in a ram cloud.

As we will discuss in the related work section, almost every element of our
system in present in some commercial system. Yet our system architecture as a
whole is unique.  The main contributions of this work are: (1) Design and
implementation of a scalable flash-based system with a global address space,
in-store computing capability and a flexible inter-controller network. (2) A
hardware-software codesign environment for incorporating user-defined in-store
processing engines. (3) Performance measurements that show the advantage of such
an architecture over using flash as a drop-in replacement for disks. (4)
Demonstration of a complex data analytics appliance which is much cheaper and
consumes an order of magnitude less power than the cloud-based alternative.

The rest of the paper is organized as follows: In section~\ref{sec:related} we
explore some existing research related to our system. In
section~\ref{sec:architecture} we describe the architecture of our rack-level
system, and in section~\ref{sec:software} we describe the software interface
that can be used to access flash and the accelerators. In
section~\ref{sec:implementation} we describe our hardware implementation of
FlashBoost, and show our results from the implementation in
section~\ref{sec:results}.  In section~\ref{sec:acceleration} we describe and
evaluate some example accelerators we have built for the FlashBoost system. 



