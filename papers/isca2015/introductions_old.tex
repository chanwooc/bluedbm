\section{Introduction}

Complex analytics of so-called "Big Data" has shown to provide extremely deep
insight into real world phenomena, but the sheer size of datasets involved makes
designing efficient analytic systems challenging. The entire working set should
ideally fit inside the collective DRAM of a cluster in order to avoid the
prohibitively high access latency of mechanical disk storage. However, such a
system quickly becomes prohibitively expensive, both in purchasing price and
power consumption. As the number of machines increase, scalability also becomes
an issue.

%	Flash storage may provide good balance between price power performance, but
%	performance bottlenecked by architecture assuming slow storage.
With the advancement of high bandwidth, low latency flash
storage technology, flash-based systems present an alternative system
configuration, as well as new architectural challenges. The access latency of
modern flash storage devices are on par with commodity networking
technologies, making a system with an affordable number of nodes that fit the working
set in collective flash storage a possible alternative to a prohibitively large cluster that
attempts to fit everything in memory. 
In such a configuration, the bottleneck shifts from the storage device to
other components such as the network latency, software overhead and bandwidth of
the host-storage interface. In order to further improve the performance,
we must move beyond point component improvements and explore cross-layer
optimizations and architectural solutions.

%	Accelerators, invocation latency limits usage scenarios, bump in the wire good.
%	In Store Processing Engine
A prominent architectural approach to further improve the performance of flash
based systems is to embed In-Store Processing (ISP) engines on the storage
device itself, in order to offload data processing and to reduce data transfer
overhead between the host and storage. In-store processing engines have multiple
desirable characteristics. It has a very low latency access to data on the
storage as it does not need to be suffer the transportation overhead to the
host CPU, and therefore also does not effected by the bandwidth limitations that
may be imposed by such connections. For example, if the host-storage link has a
lower bandwidth than the storage device, an in-storage processor can make full
use of the device bandwidth and provide bandwidth amplification over the storage
interface by pre-processing the data and only transporting a smaller amount of
relevant information. To exploit such benefits, embedding various kinds
of computation fabric in storage is under active investigation. A reconfigurable
hardware accelerator, such as an FPGA, may be a good fit in this position due
to its power efficiency and performance.

Similarly, adding computation fabric in the form of accelerators to the network
interface is also an actively researched area, in order to offload computation
and achieve bandwidth amplification by pre-processing data that goes over the
network.

%	We present bluedbm
In this work, we present FlashBoost, a high-performance
distributed flash storage and accelerator platform for big data analytics.
A FlashBoost cluster consists of multiple servers, each with a FlashBoost
storage device plugged into its PCIe slot. A FlashBoost storage device also
connects to other FlashBoost devices over a separate sideband network, making
each device a storage device as well as a network interface.
FlashBoost also provides a combined in-storage and in-network-interface
accelerator platform. Since the combined bandwidth of flash storage and the
sideband links is likely to surpass the bandwidth of the PCIe link, bandwidth
amplification by the accelerator becomes very important, as well as eliminating
the data transport overhead between storage and network interface. 

%\textbf{Would In-Data Processing be a good name to combine the two?}

%in-storage
%processing capabilities in the form of an FPGA-based hardware accelerator. It
%also provides both low-level interface into flash storage and in-storage
%processors along with a
%high-level file system interface, to make it possible to explore thin software
%layers optimized to applications.

%	BlueDBM aims to achieve the following goals
\begin{comment}
FlashBoost aims to achieve the following goals:
\begin{enumerate}
\item \textbf{High Performance}: Software should be able to access
data in the entire cluster of flash storage with with low latency and high bandwidth.
\item \textbf{Effective Hardware Acceleration}: It should provide a platform for
building hardware accelerators with very low latency access to storage and
network, in order to improve computation performance and to make maximum use of
flash and network performance.
\item \textbf{Ease of Accelerator Development}: It should provide easy-to-use
interface into system components, allowing other developers to easily
build their own hardware accelerators.
\item \textbf{Sandbox for Software Stack Development}: Provide an easy-to use
software interface into flash storage and in-storage processor access, so that the software stack
can be optimized to the specific application.
\end{enumerate}
\end{comment}

%	High level description of BlueDBM
FlashBoost follows after FPGA-accelerated distributed flash-based storage
systems such as BlueDBM~\cite{bluedbm}. Storage systems such as BlueDBM and
FlashBoost differ from traditional distributed storage systems largely in two
aspects: First, the flash storage controller includes a reconfigurable hardware
fabric, which can be used to program application-specific hardware accelerators
with low latency access into flash storage. Second: the storage
controllers and in-store processing engines are connected to each other via a separate network.
This storage device network is entirely implemented in hardware and stripped of
unnecessary software overhead, allowing very low latency communication between
storage nodes. However, the system described in BlueDBM was too small and slow
to accurately evaluate its benefits with real-world problems. Our experience
with building FlashBoost has shown that many engineering trade-offs change when
building a modern-scale system.

%	The key contribution

%	We built a machine
In order to demonstrate the performance of our architecture, we have constructed
a 20-node FlashBoost cluster with 20TBs of flash storage. Each node consists of
a host server with a 12-core Xeon server and 48GBs of DRAM, and a FlashBoost
storage device built using an off-the-shelf Xilinx VC707 FPGA development board
and two custom-designed flash boards, each with 0.5TBs of flash. All nodes are
connected to each other using both Ethernet between host servers and 8
high-speed serial links directly between the FPGAs. The FPGA communicates with
the host server over high-speed PCIe link, and host software can access the
flash storage and the in-store processing engine over either a RPC-like raw
interface or a file system interface that we have developed. A pair of our flash
boards deliver 3.2GB/s of read bandwidth, and 2.4GB/s of write bandwidth.  The 8
serial links between FPGAs each run at 10Gb/s, and all nodes act as a switch as
well as an endpoint, allowing freedom to explore topologies.

%	Rest of the paper is...
The rest of the paper is organized as follows:



\begin{comment}
As computation and communication technologies become increasingly advanced and
widespread, an unprecedented amount of digital data has become available to be
analyzed. In-depth analysis of so-called "Big Data" has shown to provide
extremely deep insight into real world phenomena.

Due to the sheer size of the datasets involved in Big Data analytics, storing
all data in fast DRAM of a cluster of machines quickly too expensive, or not
scalable. With the advancement of fast, high-capacity flash storage devices,
flash-based analytics platforms is becoming an increasingly viable alternative
to the traditional DRAM-based analytics platforms. Using flash storage, a large
dataset can fit in a smaller cluster with less machines.

Flash device's low-latency high-bandwidth characteristics move the bottleneck
away from the traditional location of secondary storage access, to other system
components such as the storage or network management software stack, or the
storage access interface between the storage and host. In order to make maximum
use of flash storage, the system architecture needs to be modified to better
balance the various components of the system.
%TODO explain in depth in Motivation: PCIe bottleneck, Ethernet bottleneck

One way to improve processing capabilities of flash storage based
systems is to use reconfigurable hardware accelerators on the storage device
itself for in-storage processing. There are multiple benefits to such a
configuration: First, an in-storage processing core can have a much lower latency
access to data. Second, operations such as filtering can be offloaded to a
in-storage processor, reducing the overhead of transporting data to the
CPU. Finally, application-specific accelerators can help maintain processing
performance while reducing the number of processing nodes.

Another way to improve performance could be to create a separate sideband
network between the near-data processors. Such a specialized network will
not require most of the bulk that comes with general-purpose network and will
facilitate an even lower-latency access into remote storage and near-data
computation resources.

In this paper, we present BlueDBM, a flash-based Big Data analytics platform to
demonstrate the benefits of such architectural solutions. BlueDBM aims to make
maximum use of flash device characteristics, in order to improve data processing
capabilities while lowering cost. To further demonstrate the capabilities of the
BlueDBM system, we are implementing multiple real-world
applications on this platform, including an SQL DBMS with hardware-accelerated
query processing, key-value store, sparse linear algebra processor and a
flash-based Memcached implementation.
\end{comment}

\begin{comment}

Due to the sheer size of the datasets involved in Big Data analytics,
transporting data to where it can be processed becomes a first order concern. In
traditional computer systems, this meant first trying to fit as much data as
possible on local DRAM, and when this becomes not viable, creating a cluster of
machines connected by a fast network fabric. Storing data in secondary storage
is done only as a last attempt when  to bring price down by sacrificing performance.
This is due to the disparity of performance across system components, where the
access latency of mechanical disk storages dwarfs all other components of a
system.

%However, traditional computing machines are not designed to 

%More data means more value <- as much data as can process.

%Big Data by definition is data too large to be processed with conventional
%methods and machines.

This landscape is changing with the widespread adoption of flash storage. Flash
storage's random access latency is multiple orders of magnitude lower than disk,
on par with most widespread networking platforms. As a result, a computing
system with fewer nodes storing data on a local flash storage could be a viable
alternative to a system with a greater number of nodes attempting to store most
data in DRAM.
\end{comment}

