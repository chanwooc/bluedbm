\section{Acceleration Performance Evaluation}
\label{sec:results_acceleration}

\begin{figure*}[ht]
\centering
\vspace{0pt}
\begin{minipage}[c]{.3\textwidth}
	\includegraphics[width=0.27\paperwidth]{graphs/obj/hammingfull-crop.pdf}
	\caption{Nearest Neighbor with FlashBoost up to Two Nodes}
	\label{fig:result_hammingfull}
\end{minipage}\hfill
\vspace{0pt}
\begin{minipage}[c]{.3\textwidth}
	\includegraphics[width=0.27\paperwidth]{graphs/obj/hamming-crop.pdf}
	\caption{Nearest Neighbor with Single-Node Throttled FlashBoost}
	\label{fig:result_hamming}
\end{minipage}\hfill
\vspace{0pt}
\begin{minipage}[c]{.3\textwidth}
	\includegraphics[width=0.27\paperwidth]{graphs/obj/graph-crop.pdf}
	\caption{Graph Traversal Performance}
	\label{fig:result_graph}
\end{minipage}
\end{figure*}

\subsection{Nearest Neighbor Search}

Figure~\ref{fig:result_hammingfull} shows the performance of a FlashBoost node
with an in-store processor against a DRAM-based machine running the same
workload. Compared to a DRAM implementation. FlashBoost has enough internal
bandwidth in the storage device that a single node of FlashBoost
outperforms DRAM up to 4 threads. It would require two FlashBoost nodes to match
the prformance of one 16-core RAM-based machine.

In order to demonstrate the benefits of the architecture, we show another
experiment where a FlashBoost node was bandwidth-throttled to the level of
commodity flash storage devices. FlashBoost was throttled to 600MB/s, which is
the bandwidth of the SATA 3.0 specification.
Figure~\ref{fig:result_hamming} shows the performance of nearest-neighbor search
with various data sources, normalized to the in-storage processing performance.
We compared FlashBoost against a high-cost fully DRAM configuration, as well as
realistic systems where some data cannot fit in DRAM.
Table~\ref{tab:nearest_neighbor} describes the benchmarks depicted in
Figure~\ref{fig:result_hamming}.

\begin{tabular}{l | p{0.25\paperwidth}}
\label{tab:nearest_neighbor}
Name & Description \\
\hline \hline
DRAM & Store all data in DRAM \\
ISP & Process data in in-storage accelerator \\
FlashBoost+SW & Use FlashBoost as raw storage \\
Seq Flash & All requests are sequential flash accesses \\
10\% Flash & Store most data in DRAM. 10\% chance of hitting flash \\
5\% Disk & Store most data in DRAM. 5\% chance of hitting disk \\
Full Flash & All requests go to flash \\
\hline
\end{tabular}

It can be seen that streaming data directly from DRAM is obviously the fastest,
and scales linearly with thread count because with DRAM bandwidth, it becomes a
computation-bound workload. The configuration that uses an in-storage processor
to offload computation is consistently faster than the software implementation,
because there is no software overhead involved, and the in-storage processor can
process the data at wire speed. Since sequential flash access with two threads
outperforms FlashBoost, it can be seen that this performance difference is not
because of the flash device performance but because of architectural differences
and better optimized software. The reason random access into commodity flash
compared to sequential access show such performance is mosty likely because it
was optimized for a non-random access pattern. FlashBoost does not have this
problem.

Using the full bandwidth of the storage system would
have made this gap even more pronounced, as the software's bandwidth would be
limited by the PCIe running at 1.6GB/s. It can be seen that when even most of
the data can fit in DRAM, even rare access into storage can have a significant
impact on performance. These results further reinforces our claim that better
storage systems are required for effective analytics of very large datasets.


\subsection{Graph Traversal}

Figure~\ref{fig:result_graph} shows the performance of distributed graph traversal in
different configurations, normalized against the graph traversal
accelerator. Table~\ref{tab:graph} describes the benchmarks depitcted in
Figure~\ref{fig:result_graph}. Data was either read from distributed flash,
distributed DRAM, or a mixture of both. All experiments were made using the same
inter-controller network either as the integrated storage network or as a
separate network interface. This is for the sake of fair comparison because  the
inter-controller network has much better performance than Ethernet.

\begin{tabular}{l | p{0.25\paperwidth}}
\label{tab:graph}
Name & Description \\
\hline \hline
ISP & In-store processor requests data from remote storage over integrated network \\
SW & Software requests data from remote storage over integrated network \\
Flash & Software requests data from remote software to read from flash \\
50\% & Store requests data from remote software. 50\% chance of hitting flash \\
70\% & Store requests data from remote software. 30\% chance of hitting flash \\
DRAM & Software requests data from remote software. Data read from DRAM \\
\hline
\end{tabular}

The performance difference between \emph{SW} and \emph{Flash} illustrates the
benefits of using the integrated network to reduce a layer of software access.
Performance of \emph{ISP} shows the benefits of reducing more software overhead
by having the ISP manage the graph traversal logic. Thanks to the reduced
latency, the graph traversal accelerator accessing distributed flash performs on
par with a more expensive system where only 30\% of the requests are serviced by
the flash.



\subsection{Hardware-Accelerate String Search}

We compared our implementation of hardware-accelerated string search running on
FlashBoost to the Linux Grep utility querying for exact string matches running
on both SSD and hard disk. Processing bandwidth and server CPU utilizations are
shown in Figure~\ref{fig:result_strstr}. We observe that the parallel MP
engines in FlashBoost are able to process a search at 1.1GB/s, which is 92\% of
the maximum sequential bandwidth a single flash board. Using FlashBoost, the
query consumes almost no CPU cycles on the host server since the query is
entirely offloaded and only the location of matched strings are returned, which
we assume is a tiny fraction of the file (0.01\% is used in our experiments).
This is 7.5x faster than software string search (Grep) on hard disks, which is
I/O bound by disk bandwidth and consumes 13\% CPU. On SSD, software string
search remains I/O bound by the storage device, but CPU utilization increases
significantly to 65\% even for this type of simple streaming compare operation.
This high utilization is problematic because string search is often only a small portion 
of more complex analytic queries that can quickly become compute bound.  As we
have shown in the results, FlashBoost can effectively alleviate this by
offloading search to the in-store processor thereby freeing up the server CPU
for other tasks. 

\begin{figure}[b]
	\centering
	\includegraphics[width=0.35\paperwidth]{graphs/obj/strstr-crop.pdf}
	\caption{String Search Bandwidth and CPU Utilization}
	\label{fig:result_strstr}
\end{figure}
 
