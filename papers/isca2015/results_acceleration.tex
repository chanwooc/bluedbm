\section{Acceleration Performance Evaluation}
\label{sec:results_acceleration}

\begin{figure*}[ht]
\centering
\vspace{0pt}
\begin{minipage}[c]{.3\textwidth}
	\includegraphics[width=0.25\paperwidth]{graphs/obj/hamming-crop.pdf}
	\caption{Nearest Neighbor with Throttled FlashBoost}
	\label{fig:result_hamming}
\end{minipage}\hfill
\vspace{0pt}
\begin{minipage}[c]{.3\textwidth}
	\includegraphics[width=0.25\paperwidth]{graphs/obj/hammingfull-crop.pdf}
	\caption{Nearest Neighbor with Full FlashBoost}
	\label{fig:result_hammingfull}
\end{minipage}\hfill
\vspace{0pt}
\begin{minipage}[c]{.3\textwidth}
	\includegraphics[width=0.25\paperwidth]{graphs/obj/graph-crop.pdf}
	\caption{Graph Traversal Performance}
	\label{fig:result_graph}
\end{minipage}
\end{figure*}

\subsection{Nearest Neighbor Search}

Figure~\ref{fig:result_hamming} shows the performance of nearest-neighbor search
with various data sources, normalized to the in-storage processing performance.
We compared FlashBoost against a high-cost fully DRAM configuration, as well as
realistic systems where some data cannot fit in DRAM.
Table~\ref{tab:nearest_neighbor} describes the benchmarks depicted in
Figure~\ref{fig:result_hamming}.

It should be noted that we have throttled our flash storage throughput to
600MB/s for this experiment, which is the bandwidth of the SATA 3.0
specification. This is to compare only the benefits of the in-store processing
architecture against other designs, otherwise the high bandwidth of the
FlashBoost hardware will result in an unfair comparison. When using all of the
2.4GB/s of our flash bandwidth, FlashBoost with ISP outperforms DRAM up to 4 threads.

\begin{tabular}{l | p{0.25\paperwidth}}
\label{tab:nearest_neighbor}
Name & Description \\
\hline \hline
ISP & Process data in in-storage accelerator \\
FlashBoost+SW & Use FlashBoost as raw storage \\
DRAM & Store all data in DRAM \\
10\% Flash & Store most data in DRAM. 10\% chance of hitting flash \\
5\% Disk & Store most data in DRAM. 5\% chance of hitting disk \\
\hline
\end{tabular}

It can be seen that streaming data directly from DRAM is obviously the fastest, and
scales linearly with thread count because it becomes a computation-bound
workload. The configuration that uses an in-storage processor to offload
computation is consistently faster than the software implementation, because
there is no software overhead involved, and the in-storage processor can process
the data at wire speed. Using the full bandwidth of the storage system would
have made this gap even more pronounced, as the software's bandwidth would be
limited by the PCIe running at 1.6GB/s. It can be seen that when even most of
the data can fit in DRAM, even rare access into storage can have a significant
impact on performance. These results further reinforces our claim that better
storage systems are required for effective analytics of very large datasets.


\subsection{Graph Traversal}

Figure~\ref{fig:result_graph} shows the performance of the graph traversal
accelerator, compared to a software implementation that accesses remote nodes
over a separate network. It can be seen that the low latency access to flash
becomes beneficial.




\subsection{Hardware-Accelerate String Search}

We were able to saturate the bandwidth of the flash store using multiple string
search cores.

This section is being written by Ming and Sungjin

