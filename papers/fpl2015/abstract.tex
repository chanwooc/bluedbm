\begin{abstract}
%\boldmath
We present a transport-layer network that aids developers in building safe,
high-performance distributed FPGA applications. Two essential features of such a
network are virtual channels and end-to-end flow control. Since different
virtual channels can have vastly different traffic patterns, a proper network
design requires flexibility in setting buffer sizes and flow control credits. In
addition, the protocol must have very low latency and low memory resource
requirements, because the communication links between FPGAs have very low
latency, and FPGAs have limited on-chip memory. These resource requirements make
protocols such as TCP/IP unsuitable in this environment. Our network implements
these features, and takes advantage of the low error characteristic of a rack
level network deployment to implement a low overhead credit based end-to-end
flow control. Our design has many parameters in the source code which can be set
at the time of FPGA synthesis. 

Our prototype cluster, which is composed of 20 Xilinx VC707 boards, each with 4
20Gb/s serial links, achieves effective bandwidth of 85\% of the maximum
physical bandwidth, and a latency of 0.5us per hop.  Our network exposes a
variable width FIFO channel abstraction, with the ability to adjust buffer size
and flow control credits per channel. Several applications have already been
developed using this network. The user feedback suggest that these features make
application development significantly easier.

%Many modern large-scale data-intensive applications can benefit from the use of
%distributed FPGAs, which offer high performance and low power consumption.
%Networking a cluster of FPGAs using generic interconnect technologies such as Ethernet and
%TCP/IP is usually difficult because of the high resource and management overhead
%of such technologies. Instead, they are often networked using a low overhead link
%level protocol using the on-chip multi-gigabit serial transceivers. Due to the
%engineering cost and on-chip resource limitations, most
%existing network implementations using these links provide a very low level
%interface, providing packet routing but rarely higher level features such as
%virtual channels or end-to-end flow control. Developing a correct distributed
%application using such a network interface is not easy for most developers, and
%this is one of the major hurdles of large scale distributed FPGA application
%development.
%
%This paper presents a transport-layer network infrastructure that aids
%application developers to build safe, high-performance distributed FPGA
%applications. The network implements convenient features such as virtual
%channels, and takes advantage of the low error characteristic of a rack level
%network deployment to implement a low overhead credit based end-to-end flow
%control. The network is parameterized so that each virtual channel can have
%different, optimized flow control characteristics. 
%Our prototype implementation achieves an effective
%bandwidth of 17Gb/s per link, which is 85\% of the maximum physical bandwidth, and a
%latency of 0.5us.  Out network exposes a variable width FIFO abstraction, which
%is convenient for application developers.

\end{abstract}

